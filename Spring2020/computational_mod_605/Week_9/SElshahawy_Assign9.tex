\documentclass[]{article}
\usepackage{lmodern}
\usepackage{amssymb,amsmath}
\usepackage{ifxetex,ifluatex}
\usepackage{fixltx2e} % provides \textsubscript
\ifnum 0\ifxetex 1\fi\ifluatex 1\fi=0 % if pdftex
  \usepackage[T1]{fontenc}
  \usepackage[utf8]{inputenc}
\else % if luatex or xelatex
  \ifxetex
    \usepackage{mathspec}
  \else
    \usepackage{fontspec}
  \fi
  \defaultfontfeatures{Ligatures=TeX,Scale=MatchLowercase}
\fi
% use upquote if available, for straight quotes in verbatim environments
\IfFileExists{upquote.sty}{\usepackage{upquote}}{}
% use microtype if available
\IfFileExists{microtype.sty}{%
\usepackage{microtype}
\UseMicrotypeSet[protrusion]{basicmath} % disable protrusion for tt fonts
}{}
\usepackage[margin=1in]{geometry}
\usepackage{hyperref}
\hypersetup{unicode=true,
            pdftitle={SElshahawy\_Assig9},
            pdfauthor={Salma Elshahawy},
            pdfborder={0 0 0},
            breaklinks=true}
\urlstyle{same}  % don't use monospace font for urls
\usepackage{color}
\usepackage{fancyvrb}
\newcommand{\VerbBar}{|}
\newcommand{\VERB}{\Verb[commandchars=\\\{\}]}
\DefineVerbatimEnvironment{Highlighting}{Verbatim}{commandchars=\\\{\}}
% Add ',fontsize=\small' for more characters per line
\usepackage{framed}
\definecolor{shadecolor}{RGB}{248,248,248}
\newenvironment{Shaded}{\begin{snugshade}}{\end{snugshade}}
\newcommand{\AlertTok}[1]{\textcolor[rgb]{0.94,0.16,0.16}{#1}}
\newcommand{\AnnotationTok}[1]{\textcolor[rgb]{0.56,0.35,0.01}{\textbf{\textit{#1}}}}
\newcommand{\AttributeTok}[1]{\textcolor[rgb]{0.77,0.63,0.00}{#1}}
\newcommand{\BaseNTok}[1]{\textcolor[rgb]{0.00,0.00,0.81}{#1}}
\newcommand{\BuiltInTok}[1]{#1}
\newcommand{\CharTok}[1]{\textcolor[rgb]{0.31,0.60,0.02}{#1}}
\newcommand{\CommentTok}[1]{\textcolor[rgb]{0.56,0.35,0.01}{\textit{#1}}}
\newcommand{\CommentVarTok}[1]{\textcolor[rgb]{0.56,0.35,0.01}{\textbf{\textit{#1}}}}
\newcommand{\ConstantTok}[1]{\textcolor[rgb]{0.00,0.00,0.00}{#1}}
\newcommand{\ControlFlowTok}[1]{\textcolor[rgb]{0.13,0.29,0.53}{\textbf{#1}}}
\newcommand{\DataTypeTok}[1]{\textcolor[rgb]{0.13,0.29,0.53}{#1}}
\newcommand{\DecValTok}[1]{\textcolor[rgb]{0.00,0.00,0.81}{#1}}
\newcommand{\DocumentationTok}[1]{\textcolor[rgb]{0.56,0.35,0.01}{\textbf{\textit{#1}}}}
\newcommand{\ErrorTok}[1]{\textcolor[rgb]{0.64,0.00,0.00}{\textbf{#1}}}
\newcommand{\ExtensionTok}[1]{#1}
\newcommand{\FloatTok}[1]{\textcolor[rgb]{0.00,0.00,0.81}{#1}}
\newcommand{\FunctionTok}[1]{\textcolor[rgb]{0.00,0.00,0.00}{#1}}
\newcommand{\ImportTok}[1]{#1}
\newcommand{\InformationTok}[1]{\textcolor[rgb]{0.56,0.35,0.01}{\textbf{\textit{#1}}}}
\newcommand{\KeywordTok}[1]{\textcolor[rgb]{0.13,0.29,0.53}{\textbf{#1}}}
\newcommand{\NormalTok}[1]{#1}
\newcommand{\OperatorTok}[1]{\textcolor[rgb]{0.81,0.36,0.00}{\textbf{#1}}}
\newcommand{\OtherTok}[1]{\textcolor[rgb]{0.56,0.35,0.01}{#1}}
\newcommand{\PreprocessorTok}[1]{\textcolor[rgb]{0.56,0.35,0.01}{\textit{#1}}}
\newcommand{\RegionMarkerTok}[1]{#1}
\newcommand{\SpecialCharTok}[1]{\textcolor[rgb]{0.00,0.00,0.00}{#1}}
\newcommand{\SpecialStringTok}[1]{\textcolor[rgb]{0.31,0.60,0.02}{#1}}
\newcommand{\StringTok}[1]{\textcolor[rgb]{0.31,0.60,0.02}{#1}}
\newcommand{\VariableTok}[1]{\textcolor[rgb]{0.00,0.00,0.00}{#1}}
\newcommand{\VerbatimStringTok}[1]{\textcolor[rgb]{0.31,0.60,0.02}{#1}}
\newcommand{\WarningTok}[1]{\textcolor[rgb]{0.56,0.35,0.01}{\textbf{\textit{#1}}}}
\usepackage{graphicx,grffile}
\makeatletter
\def\maxwidth{\ifdim\Gin@nat@width>\linewidth\linewidth\else\Gin@nat@width\fi}
\def\maxheight{\ifdim\Gin@nat@height>\textheight\textheight\else\Gin@nat@height\fi}
\makeatother
% Scale images if necessary, so that they will not overflow the page
% margins by default, and it is still possible to overwrite the defaults
% using explicit options in \includegraphics[width, height, ...]{}
\setkeys{Gin}{width=\maxwidth,height=\maxheight,keepaspectratio}
\IfFileExists{parskip.sty}{%
\usepackage{parskip}
}{% else
\setlength{\parindent}{0pt}
\setlength{\parskip}{6pt plus 2pt minus 1pt}
}
\setlength{\emergencystretch}{3em}  % prevent overfull lines
\providecommand{\tightlist}{%
  \setlength{\itemsep}{0pt}\setlength{\parskip}{0pt}}
\setcounter{secnumdepth}{0}
% Redefines (sub)paragraphs to behave more like sections
\ifx\paragraph\undefined\else
\let\oldparagraph\paragraph
\renewcommand{\paragraph}[1]{\oldparagraph{#1}\mbox{}}
\fi
\ifx\subparagraph\undefined\else
\let\oldsubparagraph\subparagraph
\renewcommand{\subparagraph}[1]{\oldsubparagraph{#1}\mbox{}}
\fi

%%% Use protect on footnotes to avoid problems with footnotes in titles
\let\rmarkdownfootnote\footnote%
\def\footnote{\protect\rmarkdownfootnote}

%%% Change title format to be more compact
\usepackage{titling}

% Create subtitle command for use in maketitle
\providecommand{\subtitle}[1]{
  \posttitle{
    \begin{center}\large#1\end{center}
    }
}

\setlength{\droptitle}{-2em}

  \title{SElshahawy\_Assig9}
    \pretitle{\vspace{\droptitle}\centering\huge}
  \posttitle{\par}
    \author{Salma Elshahawy}
    \preauthor{\centering\large\emph}
  \postauthor{\par}
      \predate{\centering\large\emph}
  \postdate{\par}
    \date{3/23/2020}

\usepackage{geometry}

\begin{document}
\maketitle

\hypertarget{question_11-page-363-introduction-to-probability}{%
\subsection{Question\_11, page 363, Introduction to
probability}\label{question_11-page-363-introduction-to-probability}}

The price of one share of stock in the Pilsdorff Beer Company (see Exer-
cise 8.2.12) is given by Yn on the nth day of the year. Finn observes
that the differences Xn = Yn+1 − Yn appear to be independent random
variables with a common distribution having mean μ = 0 and variance σ2 =
1/4. If Y1 = 100, estimate the probability that Y365 is (a) \(\ge 100\)
(b) \(\ge 110\) (c) \(\ge 120\)

\textbf{Answer:-}

The summation of random variables will be as the following:

\[{ S }_{ n }\quad =\quad \sum _{ x\quad =\quad 1 }^{ x\quad =\quad n }{ { X }_{ 1 } } +\quad { X }_{ 2 }\quad +\quad { X }_{ 3 }\quad +\quad ......\quad +\quad { X }_{ n }\]
Now substitute X by \({ Y }_{ n+1 }\quad -\quad { Y }_{ n }\)

\[{ S }_{ n }\quad =\quad \sum { \left( { Y }_{ 2 }\quad -\quad { Y }_{ 1 } \right)  } \quad +\quad \left( { Y }_{ 3 }\quad -\quad { Y }_{ 2 } \right) \quad +\quad ......\quad +\quad \left( { Y }_{ n+1 }\quad -\quad { Y }_{ n } \right) \]
Then we end up with:

\[{ S }_{ n }\quad =\quad { Y }_{ n+1 }\quad -\quad { Y }_{ n }\]

\[{ S }_{ n }\quad =\quad { Y }_{ n+1 }\quad -\quad 100\quad (where\quad { Y }_{ n }\quad =\quad 100)\]
Now we have to get the mean and standard deviation
\[var{ S }_{ n }\quad =\quad { \sigma  }^{ 2 }{ S }_{ n }\quad =\quad \frac { 1 }{ 4 } \times \quad \sum _{ X=0 }^{ n }{ { { \sigma  } }_{ X }^{ 2 } } \quad =\quad \frac { n }{ 4 }\]
\[SD{ S }_{ n }\quad =\quad \sqrt { \frac { n }{ 4 }  } \quad =\quad \frac { \sqrt { n }  }{ 2 } \]

Consider n = 364 where \({ n+1 } = 365\)

\[{ S }_{ 364 }\quad =\quad { Y }_{ 365 }\quad -\quad 100\\ { Y }_{ 365 }\quad =\quad { S }_{ 364 }\quad +\quad 100\]
calculate variance for n = 364

\begin{Shaded}
\begin{Highlighting}[]
\NormalTok{n =}\StringTok{ }\DecValTok{364}
\NormalTok{var_s_}\DecValTok{364}\NormalTok{ =}\StringTok{ }\NormalTok{n}\OperatorTok{/}\DecValTok{4}
\NormalTok{var_s_}\DecValTok{364}
\end{Highlighting}
\end{Shaded}

\begin{verbatim}
## [1] 91
\end{verbatim}

\begin{Shaded}
\begin{Highlighting}[]
\NormalTok{sd_s_}\DecValTok{364}\NormalTok{ =}\StringTok{ }\KeywordTok{sqrt}\NormalTok{(n)}\OperatorTok{/}\DecValTok{2}
\NormalTok{sd_s_}\DecValTok{364}
\end{Highlighting}
\end{Shaded}

\begin{verbatim}
## [1] 9.539392
\end{verbatim}

\hypertarget{a-ge-100}{%
\subsubsection{\texorpdfstring{(a)
\(\ge 100\)}{(a) \textbackslash ge 100}}\label{a-ge-100}}

\begin{Shaded}
\begin{Highlighting}[]
\NormalTok{z =}\StringTok{ }\NormalTok{(}\DecValTok{100-100}\NormalTok{) }\OperatorTok{/}\StringTok{ }\NormalTok{sd_s_}\DecValTok{364}
\KeywordTok{pnorm}\NormalTok{(z, }\DataTypeTok{lower.tail =} \OtherTok{FALSE}\NormalTok{)}
\end{Highlighting}
\end{Shaded}

\begin{verbatim}
## [1] 0.5
\end{verbatim}

\hypertarget{b-ge-110}{%
\subsubsection{\texorpdfstring{(b)
\(\ge 110\)}{(b) \textbackslash ge 110}}\label{b-ge-110}}

\begin{Shaded}
\begin{Highlighting}[]
\NormalTok{z =}\StringTok{ }\NormalTok{(}\DecValTok{110-100}\NormalTok{) }\OperatorTok{/}\StringTok{ }\NormalTok{sd_s_}\DecValTok{364}
\KeywordTok{pnorm}\NormalTok{(z, }\DataTypeTok{lower.tail =} \OtherTok{FALSE}\NormalTok{)}
\end{Highlighting}
\end{Shaded}

\begin{verbatim}
## [1] 0.1472537
\end{verbatim}

\hypertarget{c-ge-120}{%
\subsubsection{\texorpdfstring{(c)
\(\ge 120\)}{(c) \textbackslash ge 120}}\label{c-ge-120}}

\begin{Shaded}
\begin{Highlighting}[]
\NormalTok{z =}\StringTok{ }\NormalTok{(}\DecValTok{120-100}\NormalTok{) }\OperatorTok{/}\StringTok{ }\NormalTok{sd_s_}\DecValTok{364}
\KeywordTok{pnorm}\NormalTok{(z, }\DataTypeTok{lower.tail =} \OtherTok{FALSE}\NormalTok{)}
\end{Highlighting}
\end{Shaded}

\begin{verbatim}
## [1] 0.01801584
\end{verbatim}

\hypertarget{question_2}{%
\subsection{Question\_2}\label{question_2}}

Calculate the expected value and variance of the binomial distribution
using the moment generating function.

\textbf{Answer:-}

I will start with the definition of
\({ M }_{ x }(t)\quad =\quad E({ e }^{ tx })\) for a descrete variables:
\[{ M }_{ x }(t)\quad =\quad \sum _{ x=0 }^{ n }{ { e }^{ tx } } \cdot \quad p(x)\quad \longrightarrow \quad 1\]
from the binomial theory:
\[p(x)\quad =\quad \begin{pmatrix} n \\ x \end{pmatrix}\cdot { \quad p }^{ x }\cdot \quad { (1-p) }^{ n-x }\quad \rightarrow \quad 2\]
by substitute equation 2 into equation 1 we get:
\[{ M }_{ x }(t)\quad =\quad \sum _{ x=0 }^{ n }{ { e }^{ tx } } \cdot \quad \begin{pmatrix} n \\ x \end{pmatrix}\cdot { \quad p }^{ x }\cdot \quad { (1-p) }^{ n-x }\]
take e and p the same brackets to the power of x we get:
\[{ M }_{ x }(t)\quad =\quad \sum _{ x=0 }^{ n }{ \begin{pmatrix} n \\ x \end{pmatrix} } \cdot \quad ({ e }^{ t }{ \quad p) }^{ x }\cdot \quad { (1-p) }^{ n-x }\quad \rightarrow \quad 3\]
and we know from the Binomial theorm that
\[\sum _{ x=0 }^{ n }{ \begin{pmatrix} n \\ x \end{pmatrix} } { \quad y }^{ x }\quad { \quad z }^{ n-x }\quad =\quad { (y\quad +\quad z) }^{ n }\quad for\quad any\quad x,y,z\quad \epsilon \quad R\quad \quad \rightarrow \quad 4\]

Then we can use the Binomial theorm to get the MGF from equation 3 then
we get:

\[ { M }_{ x }(t)\quad =\quad { (({ e }^{ t }{ \quad p) }\quad +\quad (1-p)) }^{ n }\quad for\quad t\quad \epsilon \quad R\quad \quad \rightarrow \quad done(the\quad expected\quad value)\]

\[var(x) = second moment - { first\quad moment }^{ 2 }\quad at\quad t\quad =\quad 0\]
\[{ M }_{ x }^{ ' }(t)\quad =\quad n{ (({ e }^{ t }{ \quad p) }\quad +\quad (1-p)) }^{ n-1 }\cdot \quad { e }^{ t }{ \quad p }\quad \rightarrow \quad 1st\quad moment\]
\[{ M }_{ x }^{ ' }(0)\quad =\quad np\quad \rightarrow \quad { { M }_{ x }^{ ' }(0) }^{ 2 }\quad =\quad { n }^{ 2 }{ p }^{ 2 }\quad \rightarrow \quad 5\]
\[{ M }_{ x }^{ '' }(t)\quad =\quad n\cdot (n-1)\cdot { (({ e }^{ t }{ \quad p) }\quad +\quad (1-p)) }^{ n-2 }\cdot \quad { e }^{ 2t }{ \quad p }^{ 2 }\quad +\quad { (({ e }^{ t }{ \quad p) }\quad +\quad (1-p)) }^{ n-1 }\quad \cdot \quad n\quad \cdot \quad { e }^{ t }{ \quad p }\quad \rightarrow \quad 2nd\quad moment\]
\[{ M }_{ x }^{ '' }(0)\quad =\quad n\cdot (n-1){ \quad p }^{ 2 }\quad +\quad np\quad \rightarrow \quad 6\]

by subtracting equation 5 from 6 we get variance

\[var(x)\quad =\quad n\cdot (n-1){ \quad p }^{ 2 }\quad +\quad np\quad -\quad { n }^{ 2 }{ p }^{ 2 }\]
\[var(x)\quad =\quad np(1-p)\quad \rightarrow \quad done\quad (the\quad variance)\]

\hypertarget{question_3}{%
\subsection{Question\_3}\label{question_3}}

Calculate the expected value and variance of the exponential
distribution using the moment generating function.

\textbf{Answer:-}

for the exponential distribution: \[f(x)=\lambda e^{-\lambda x}\] The
MGF should be:
\[{ M }_{ x }(t)=\frac{\lambda}{\lambda-t} \quad where \quad t<\lambda\]

Getting the first moment:
\[{ M }_{ x }^{ '' }(t) = \frac{\lambda}{(\lambda-t)^2}\] The second
moment: \[{ M }_{ x }^{ '' }(t) = \frac{2\lambda}{(\lambda-t)^3}\]
\[{ E }(x) = { M }_{ x }(t) \quad at \quad t = 0\]
\[{ E }(x) = \frac{1}{\lambda}\]

\[{ var }(x) = { M }_{ x }^{ '' }(t)-{ M }_{ x }^{ ' }(t)^2 \quad at \quad t = 0\]

\[{ var }(x) = \frac{1}{\lambda^2}\]


\end{document}
