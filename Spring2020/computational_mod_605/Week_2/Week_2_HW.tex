\documentclass[]{article}
\usepackage{lmodern}
\usepackage{amssymb,amsmath}
\usepackage{ifxetex,ifluatex}
\usepackage{fixltx2e} % provides \textsubscript
\ifnum 0\ifxetex 1\fi\ifluatex 1\fi=0 % if pdftex
  \usepackage[T1]{fontenc}
  \usepackage[utf8]{inputenc}
\else % if luatex or xelatex
  \ifxetex
    \usepackage{mathspec}
  \else
    \usepackage{fontspec}
  \fi
  \defaultfontfeatures{Ligatures=TeX,Scale=MatchLowercase}
\fi
% use upquote if available, for straight quotes in verbatim environments
\IfFileExists{upquote.sty}{\usepackage{upquote}}{}
% use microtype if available
\IfFileExists{microtype.sty}{%
\usepackage{microtype}
\UseMicrotypeSet[protrusion]{basicmath} % disable protrusion for tt fonts
}{}
\usepackage[margin=1in]{geometry}
\usepackage{hyperref}
\hypersetup{unicode=true,
            pdftitle={Week\_2 605 assignment},
            pdfauthor={Salma Elshahawy},
            pdfborder={0 0 0},
            breaklinks=true}
\urlstyle{same}  % don't use monospace font for urls
\usepackage{color}
\usepackage{fancyvrb}
\newcommand{\VerbBar}{|}
\newcommand{\VERB}{\Verb[commandchars=\\\{\}]}
\DefineVerbatimEnvironment{Highlighting}{Verbatim}{commandchars=\\\{\}}
% Add ',fontsize=\small' for more characters per line
\usepackage{framed}
\definecolor{shadecolor}{RGB}{248,248,248}
\newenvironment{Shaded}{\begin{snugshade}}{\end{snugshade}}
\newcommand{\AlertTok}[1]{\textcolor[rgb]{0.94,0.16,0.16}{#1}}
\newcommand{\AnnotationTok}[1]{\textcolor[rgb]{0.56,0.35,0.01}{\textbf{\textit{#1}}}}
\newcommand{\AttributeTok}[1]{\textcolor[rgb]{0.77,0.63,0.00}{#1}}
\newcommand{\BaseNTok}[1]{\textcolor[rgb]{0.00,0.00,0.81}{#1}}
\newcommand{\BuiltInTok}[1]{#1}
\newcommand{\CharTok}[1]{\textcolor[rgb]{0.31,0.60,0.02}{#1}}
\newcommand{\CommentTok}[1]{\textcolor[rgb]{0.56,0.35,0.01}{\textit{#1}}}
\newcommand{\CommentVarTok}[1]{\textcolor[rgb]{0.56,0.35,0.01}{\textbf{\textit{#1}}}}
\newcommand{\ConstantTok}[1]{\textcolor[rgb]{0.00,0.00,0.00}{#1}}
\newcommand{\ControlFlowTok}[1]{\textcolor[rgb]{0.13,0.29,0.53}{\textbf{#1}}}
\newcommand{\DataTypeTok}[1]{\textcolor[rgb]{0.13,0.29,0.53}{#1}}
\newcommand{\DecValTok}[1]{\textcolor[rgb]{0.00,0.00,0.81}{#1}}
\newcommand{\DocumentationTok}[1]{\textcolor[rgb]{0.56,0.35,0.01}{\textbf{\textit{#1}}}}
\newcommand{\ErrorTok}[1]{\textcolor[rgb]{0.64,0.00,0.00}{\textbf{#1}}}
\newcommand{\ExtensionTok}[1]{#1}
\newcommand{\FloatTok}[1]{\textcolor[rgb]{0.00,0.00,0.81}{#1}}
\newcommand{\FunctionTok}[1]{\textcolor[rgb]{0.00,0.00,0.00}{#1}}
\newcommand{\ImportTok}[1]{#1}
\newcommand{\InformationTok}[1]{\textcolor[rgb]{0.56,0.35,0.01}{\textbf{\textit{#1}}}}
\newcommand{\KeywordTok}[1]{\textcolor[rgb]{0.13,0.29,0.53}{\textbf{#1}}}
\newcommand{\NormalTok}[1]{#1}
\newcommand{\OperatorTok}[1]{\textcolor[rgb]{0.81,0.36,0.00}{\textbf{#1}}}
\newcommand{\OtherTok}[1]{\textcolor[rgb]{0.56,0.35,0.01}{#1}}
\newcommand{\PreprocessorTok}[1]{\textcolor[rgb]{0.56,0.35,0.01}{\textit{#1}}}
\newcommand{\RegionMarkerTok}[1]{#1}
\newcommand{\SpecialCharTok}[1]{\textcolor[rgb]{0.00,0.00,0.00}{#1}}
\newcommand{\SpecialStringTok}[1]{\textcolor[rgb]{0.31,0.60,0.02}{#1}}
\newcommand{\StringTok}[1]{\textcolor[rgb]{0.31,0.60,0.02}{#1}}
\newcommand{\VariableTok}[1]{\textcolor[rgb]{0.00,0.00,0.00}{#1}}
\newcommand{\VerbatimStringTok}[1]{\textcolor[rgb]{0.31,0.60,0.02}{#1}}
\newcommand{\WarningTok}[1]{\textcolor[rgb]{0.56,0.35,0.01}{\textbf{\textit{#1}}}}
\usepackage{graphicx,grffile}
\makeatletter
\def\maxwidth{\ifdim\Gin@nat@width>\linewidth\linewidth\else\Gin@nat@width\fi}
\def\maxheight{\ifdim\Gin@nat@height>\textheight\textheight\else\Gin@nat@height\fi}
\makeatother
% Scale images if necessary, so that they will not overflow the page
% margins by default, and it is still possible to overwrite the defaults
% using explicit options in \includegraphics[width, height, ...]{}
\setkeys{Gin}{width=\maxwidth,height=\maxheight,keepaspectratio}
\IfFileExists{parskip.sty}{%
\usepackage{parskip}
}{% else
\setlength{\parindent}{0pt}
\setlength{\parskip}{6pt plus 2pt minus 1pt}
}
\setlength{\emergencystretch}{3em}  % prevent overfull lines
\providecommand{\tightlist}{%
  \setlength{\itemsep}{0pt}\setlength{\parskip}{0pt}}
\setcounter{secnumdepth}{0}
% Redefines (sub)paragraphs to behave more like sections
\ifx\paragraph\undefined\else
\let\oldparagraph\paragraph
\renewcommand{\paragraph}[1]{\oldparagraph{#1}\mbox{}}
\fi
\ifx\subparagraph\undefined\else
\let\oldsubparagraph\subparagraph
\renewcommand{\subparagraph}[1]{\oldsubparagraph{#1}\mbox{}}
\fi

%%% Use protect on footnotes to avoid problems with footnotes in titles
\let\rmarkdownfootnote\footnote%
\def\footnote{\protect\rmarkdownfootnote}

%%% Change title format to be more compact
\usepackage{titling}

% Create subtitle command for use in maketitle
\providecommand{\subtitle}[1]{
  \posttitle{
    \begin{center}\large#1\end{center}
    }
}

\setlength{\droptitle}{-2em}

  \title{Week\_2 605 assignment}
    \pretitle{\vspace{\droptitle}\centering\huge}
  \posttitle{\par}
    \author{Salma Elshahawy}
    \preauthor{\centering\large\emph}
  \postauthor{\par}
      \predate{\centering\large\emph}
  \postdate{\par}
    \date{02/08/2020}

\usepackage{geometry}

\begin{document}
\maketitle

\hypertarget{problem-set_1}{%
\subsection{Problem set\_1}\label{problem-set_1}}

\hypertarget{show-that-a-t-aquad-neq-quad-a-a-t-in-general.}{%
\subsubsection{\texorpdfstring{(1) Show that
``\({ A }^{ T }A\quad \neq \quad A{ A }^{ T }\)'' in
general.}{(1) Show that ``\{ A \}\^{}\{ T \}A\textbackslash quad \textbackslash neq \textbackslash quad A\{ A \}\^{}\{ T \}'' in general.}}\label{show-that-a-t-aquad-neq-quad-a-a-t-in-general.}}

\begin{Shaded}
\begin{Highlighting}[]
\CommentTok{# creating a matrix A}
\NormalTok{A <-}\StringTok{ }\KeywordTok{matrix}\NormalTok{(}\KeywordTok{c}\NormalTok{(}\DecValTok{1}\NormalTok{,}\DecValTok{2}\NormalTok{,}\DecValTok{3}\NormalTok{,}\DecValTok{4}\NormalTok{,}\DecValTok{5}\NormalTok{,}\DecValTok{6}\NormalTok{,}\DecValTok{7}\NormalTok{,}\DecValTok{8}\NormalTok{,}\DecValTok{9}\NormalTok{), }\DataTypeTok{nrow =} \DecValTok{3}\NormalTok{)}
\NormalTok{A}
\end{Highlighting}
\end{Shaded}

\begin{verbatim}
##      [,1] [,2] [,3]
## [1,]    1    4    7
## [2,]    2    5    8
## [3,]    3    6    9
\end{verbatim}

\begin{Shaded}
\begin{Highlighting}[]
\CommentTok{# get A transpose and assign it to At}
\NormalTok{At <-}\StringTok{ }\KeywordTok{t}\NormalTok{(A)}
\NormalTok{At}
\end{Highlighting}
\end{Shaded}

\begin{verbatim}
##      [,1] [,2] [,3]
## [1,]    1    2    3
## [2,]    4    5    6
## [3,]    7    8    9
\end{verbatim}

\begin{Shaded}
\begin{Highlighting}[]
\CommentTok{# mutliply matrix A by it's transpose A*t(A)}
\NormalTok{left_side <-}\StringTok{ }\NormalTok{A }\OperatorTok\StringTok{ }\NormalTok{At}
\NormalTok{left_side}
\end{Highlighting}
\end{Shaded}

\begin{verbatim}
##      [,1] [,2] [,3]
## [1,]   66   78   90
## [2,]   78   93  108
## [3,]   90  108  126
\end{verbatim}

\begin{Shaded}
\begin{Highlighting}[]
\CommentTok{# mutliply matrix transpose A by the original matrix t(A)*A}
\NormalTok{right_side <-}\StringTok{ }\NormalTok{At }\OperatorTok\StringTok{ }\NormalTok{A}
\NormalTok{right_side}
\end{Highlighting}
\end{Shaded}

\begin{verbatim}
##      [,1] [,2] [,3]
## [1,]   14   32   50
## [2,]   32   77  122
## [3,]   50  122  194
\end{verbatim}

\begin{Shaded}
\begin{Highlighting}[]
\CommentTok{# check if the two sides are in equilibrium state}
\NormalTok{left_side }\OperatorTok{==}\StringTok{ }\NormalTok{right_side}
\end{Highlighting}
\end{Shaded}

\begin{verbatim}
##       [,1]  [,2]  [,3]
## [1,] FALSE FALSE FALSE
## [2,] FALSE FALSE FALSE
## [3,] FALSE FALSE FALSE
\end{verbatim}

Both sides are not equal, so in general
\({ A }^{ T }A\quad \neq \quad A{ A }^{ T }\)

\hypertarget{for-a-special-type-of-square-matrix-a-we-get-a-t-aquad-quad-a-a-t-.-under-what-conditions-could-this-be-true-hint-the-identity-matrix-i-is-an-example-of-such-a-matrix.}{%
\subsubsection{\texorpdfstring{(2) For a special type of square matrix
A, we get \({ A }^{ T }A\quad =\quad A{ A }^{ T }\) . Under what
conditions could this be true? (Hint: The Identity matrix I is an
example of such a
matrix).}{(2) For a special type of square matrix A, we get \{ A \}\^{}\{ T \}A\textbackslash quad =\textbackslash quad A\{ A \}\^{}\{ T \} . Under what conditions could this be true? (Hint: The Identity matrix I is an example of such a matrix).}}\label{for-a-special-type-of-square-matrix-a-we-get-a-t-aquad-quad-a-a-t-.-under-what-conditions-could-this-be-true-hint-the-identity-matrix-i-is-an-example-of-such-a-matrix.}}

\emph{Please typeset your response using LaTeX/SWeave mode in RStudio.
If you do it in paper, please either scan or take a picture of the work
and submit it. Please en- sure that your image is legible and that your
submissions are named using your first initial, last name, assignment
and problem set within the assignment.}

\begin{Shaded}
\begin{Highlighting}[]
\CommentTok{# construct a diagonal matrix}
\CommentTok{# }
\NormalTok{A <-}\StringTok{ }\KeywordTok{matrix}\NormalTok{(}\KeywordTok{c}\NormalTok{(}\DecValTok{1}\NormalTok{,}\DecValTok{0}\NormalTok{,}\DecValTok{0}\NormalTok{,}\DecValTok{0}\NormalTok{,}\DecValTok{5}\NormalTok{,}\DecValTok{0}\NormalTok{,}\DecValTok{0}\NormalTok{,}\DecValTok{0}\NormalTok{,}\DecValTok{9}\NormalTok{), }\DataTypeTok{nrow =} \DecValTok{3}\NormalTok{)}
\NormalTok{A <-}\StringTok{ }\DecValTok{5} \OperatorTok{*}\StringTok{ }\NormalTok{A }\CommentTok{# multiply by a scaler}
\NormalTok{A}
\end{Highlighting}
\end{Shaded}

\begin{verbatim}
##      [,1] [,2] [,3]
## [1,]    5    0    0
## [2,]    0   25    0
## [3,]    0    0   45
\end{verbatim}

\begin{Shaded}
\begin{Highlighting}[]
\NormalTok{A}\OperatorTok\KeywordTok{t}\NormalTok{(A) }\OperatorTok{==}\StringTok{ }\KeywordTok{t}\NormalTok{(A)}\OperatorTok\NormalTok{A}
\end{Highlighting}
\end{Shaded}

\begin{verbatim}
##      [,1] [,2] [,3]
## [1,] TRUE TRUE TRUE
## [2,] TRUE TRUE TRUE
## [3,] TRUE TRUE TRUE
\end{verbatim}

I Observed that when a matrix is symmetric, the matrix is equal to its
transpose, \({ A }^{ T }A\quad =\quad A{ A }^{ T }\)

The conditions:

1- Be a square matrix
\(A_{ i\times j }\quad =\quad { A }_{ j\times i }^{ T }\)

2- The upper and lower triangle are equal to zero
\(\begin{matrix} { a }_{ ij } & 0 & 0 \\ 0 & { a }_{ ij } & 0 \\ 0 & 0 & { a }_{ ij } \end{matrix}\)

\hypertarget{problem-set_2}{%
\subsection{Problem set\_2}\label{problem-set_2}}

Matrix factorization is a very important problem. There are
supercomputers built just to do matrix factorizations. Every second you
are on an airplane, matrices are being factorized. Radars that track
flights use a technique called Kalman filtering. At the heart of Kalman
Filtering is a Matrix Factorization operation. Kalman Filters are
solving linear systems of equations when they track your flight using
radars. Write an R function to factorize a square matrix A into LU or
LDU, whichever you prefer.

\emph{Please submit your response in an R Markdown document using our
class naming convention. You don't have to worry about permuting rows of
A and you can assume that A is less than 5x5, if you need to hard-code
any variables in your code. If you doing the entire assignment in R,
then please submit only one markdown document for both the problems.}

\begin{Shaded}
\begin{Highlighting}[]
\NormalTok{a <-}\StringTok{ }\KeywordTok{matrix}\NormalTok{(}\KeywordTok{c}\NormalTok{(}\DecValTok{2}\NormalTok{,}\DecValTok{4}\NormalTok{,}\OperatorTok{-}\DecValTok{2}\NormalTok{,}
              \DecValTok{1}\NormalTok{,}\OperatorTok{-}\DecValTok{1}\NormalTok{,}\DecValTok{5}\NormalTok{,}
              \DecValTok{3}\NormalTok{,}\DecValTok{3}\NormalTok{,}\DecValTok{5}\NormalTok{), }\DataTypeTok{nrow =} \DecValTok{3}\NormalTok{, }\DataTypeTok{byrow =} \OtherTok{TRUE}\NormalTok{)}
\NormalTok{a}
\end{Highlighting}
\end{Shaded}

\begin{verbatim}
##      [,1] [,2] [,3]
## [1,]    2    4   -2
## [2,]    1   -1    5
## [3,]    3    3    5
\end{verbatim}

\begin{Shaded}
\begin{Highlighting}[]
\NormalTok{get_UL <-}\StringTok{ }\ControlFlowTok{function}\NormalTok{(a) \{}
\NormalTok{  U =}\StringTok{ }\NormalTok{a}
\NormalTok{  L =}\StringTok{ }\KeywordTok{diag}\NormalTok{(}\KeywordTok{nrow}\NormalTok{(a))}
\NormalTok{  n =}\StringTok{ }\KeywordTok{nrow}\NormalTok{(a)}
  \ControlFlowTok{for}\NormalTok{ (i }\ControlFlowTok{in} \DecValTok{1}\OperatorTok{:}\NormalTok{n) \{}
\NormalTok{    k =}\StringTok{ }\KeywordTok{seq}\NormalTok{(}\DecValTok{2}\NormalTok{, n)}
    \ControlFlowTok{for}\NormalTok{ (j }\ControlFlowTok{in}\NormalTok{ k) \{}
      \ControlFlowTok{if}\NormalTok{(j }\OperatorTok{>}\StringTok{ }\NormalTok{i) \{}
        \CommentTok{# get the multiplier and add it to the L matrix}
\NormalTok{        s =}\StringTok{ }\NormalTok{U[[j,i]]}\OperatorTok{/}\NormalTok{U[[i,i]]}
\NormalTok{        L[j,i] =}\StringTok{ }\NormalTok{s}
        \CommentTok{# reduce by reduction and shuffle to the U matrix}
\NormalTok{        U[j,] =}\StringTok{ }\NormalTok{U[j,] }\OperatorTok{-}\StringTok{ }\NormalTok{s }\OperatorTok{*}\StringTok{ }\NormalTok{U[i,]}
\NormalTok{      \}}
\NormalTok{    \}}
\NormalTok{  \}}
  \KeywordTok{return}\NormalTok{(}\KeywordTok{list}\NormalTok{(}\DataTypeTok{U =}\NormalTok{ U, }\DataTypeTok{L =}\NormalTok{ L))}
\NormalTok{\}}

\NormalTok{value <-}\StringTok{ }\KeywordTok{get_UL}\NormalTok{(a)  }
\NormalTok{U_matrix <-}\StringTok{ }\NormalTok{value}\OperatorTok{$}\NormalTok{U}
\NormalTok{U_matrix}
\end{Highlighting}
\end{Shaded}

\begin{verbatim}
##      [,1] [,2] [,3]
## [1,]    2    4   -2
## [2,]    0   -3    6
## [3,]    0    0    2
\end{verbatim}

\begin{Shaded}
\begin{Highlighting}[]
\NormalTok{L_matrix <-}\StringTok{ }\NormalTok{value}\OperatorTok{$}\NormalTok{L}
\NormalTok{L_matrix}
\end{Highlighting}
\end{Shaded}

\begin{verbatim}
##      [,1] [,2] [,3]
## [1,]  1.0    0    0
## [2,]  0.5    1    0
## [3,]  1.5    1    1
\end{verbatim}


\end{document}
