\documentclass[]{article}
\usepackage{lmodern}
\usepackage{amssymb,amsmath}
\usepackage{ifxetex,ifluatex}
\usepackage{fixltx2e} % provides \textsubscript
\ifnum 0\ifxetex 1\fi\ifluatex 1\fi=0 % if pdftex
  \usepackage[T1]{fontenc}
  \usepackage[utf8]{inputenc}
\else % if luatex or xelatex
  \ifxetex
    \usepackage{mathspec}
  \else
    \usepackage{fontspec}
  \fi
  \defaultfontfeatures{Ligatures=TeX,Scale=MatchLowercase}
\fi
% use upquote if available, for straight quotes in verbatim environments
\IfFileExists{upquote.sty}{\usepackage{upquote}}{}
% use microtype if available
\IfFileExists{microtype.sty}{%
\usepackage{microtype}
\UseMicrotypeSet[protrusion]{basicmath} % disable protrusion for tt fonts
}{}
\usepackage[margin=1in]{geometry}
\usepackage{hyperref}
\hypersetup{unicode=true,
            pdftitle={Assign\_1(Vectors and Matrices)},
            pdfauthor={Salma Elshahawy},
            pdfborder={0 0 0},
            breaklinks=true}
\urlstyle{same}  % don't use monospace font for urls
\usepackage{color}
\usepackage{fancyvrb}
\newcommand{\VerbBar}{|}
\newcommand{\VERB}{\Verb[commandchars=\\\{\}]}
\DefineVerbatimEnvironment{Highlighting}{Verbatim}{commandchars=\\\{\}}
% Add ',fontsize=\small' for more characters per line
\usepackage{framed}
\definecolor{shadecolor}{RGB}{248,248,248}
\newenvironment{Shaded}{\begin{snugshade}}{\end{snugshade}}
\newcommand{\AlertTok}[1]{\textcolor[rgb]{0.94,0.16,0.16}{#1}}
\newcommand{\AnnotationTok}[1]{\textcolor[rgb]{0.56,0.35,0.01}{\textbf{\textit{#1}}}}
\newcommand{\AttributeTok}[1]{\textcolor[rgb]{0.77,0.63,0.00}{#1}}
\newcommand{\BaseNTok}[1]{\textcolor[rgb]{0.00,0.00,0.81}{#1}}
\newcommand{\BuiltInTok}[1]{#1}
\newcommand{\CharTok}[1]{\textcolor[rgb]{0.31,0.60,0.02}{#1}}
\newcommand{\CommentTok}[1]{\textcolor[rgb]{0.56,0.35,0.01}{\textit{#1}}}
\newcommand{\CommentVarTok}[1]{\textcolor[rgb]{0.56,0.35,0.01}{\textbf{\textit{#1}}}}
\newcommand{\ConstantTok}[1]{\textcolor[rgb]{0.00,0.00,0.00}{#1}}
\newcommand{\ControlFlowTok}[1]{\textcolor[rgb]{0.13,0.29,0.53}{\textbf{#1}}}
\newcommand{\DataTypeTok}[1]{\textcolor[rgb]{0.13,0.29,0.53}{#1}}
\newcommand{\DecValTok}[1]{\textcolor[rgb]{0.00,0.00,0.81}{#1}}
\newcommand{\DocumentationTok}[1]{\textcolor[rgb]{0.56,0.35,0.01}{\textbf{\textit{#1}}}}
\newcommand{\ErrorTok}[1]{\textcolor[rgb]{0.64,0.00,0.00}{\textbf{#1}}}
\newcommand{\ExtensionTok}[1]{#1}
\newcommand{\FloatTok}[1]{\textcolor[rgb]{0.00,0.00,0.81}{#1}}
\newcommand{\FunctionTok}[1]{\textcolor[rgb]{0.00,0.00,0.00}{#1}}
\newcommand{\ImportTok}[1]{#1}
\newcommand{\InformationTok}[1]{\textcolor[rgb]{0.56,0.35,0.01}{\textbf{\textit{#1}}}}
\newcommand{\KeywordTok}[1]{\textcolor[rgb]{0.13,0.29,0.53}{\textbf{#1}}}
\newcommand{\NormalTok}[1]{#1}
\newcommand{\OperatorTok}[1]{\textcolor[rgb]{0.81,0.36,0.00}{\textbf{#1}}}
\newcommand{\OtherTok}[1]{\textcolor[rgb]{0.56,0.35,0.01}{#1}}
\newcommand{\PreprocessorTok}[1]{\textcolor[rgb]{0.56,0.35,0.01}{\textit{#1}}}
\newcommand{\RegionMarkerTok}[1]{#1}
\newcommand{\SpecialCharTok}[1]{\textcolor[rgb]{0.00,0.00,0.00}{#1}}
\newcommand{\SpecialStringTok}[1]{\textcolor[rgb]{0.31,0.60,0.02}{#1}}
\newcommand{\StringTok}[1]{\textcolor[rgb]{0.31,0.60,0.02}{#1}}
\newcommand{\VariableTok}[1]{\textcolor[rgb]{0.00,0.00,0.00}{#1}}
\newcommand{\VerbatimStringTok}[1]{\textcolor[rgb]{0.31,0.60,0.02}{#1}}
\newcommand{\WarningTok}[1]{\textcolor[rgb]{0.56,0.35,0.01}{\textbf{\textit{#1}}}}
\usepackage{graphicx,grffile}
\makeatletter
\def\maxwidth{\ifdim\Gin@nat@width>\linewidth\linewidth\else\Gin@nat@width\fi}
\def\maxheight{\ifdim\Gin@nat@height>\textheight\textheight\else\Gin@nat@height\fi}
\makeatother
% Scale images if necessary, so that they will not overflow the page
% margins by default, and it is still possible to overwrite the defaults
% using explicit options in \includegraphics[width, height, ...]{}
\setkeys{Gin}{width=\maxwidth,height=\maxheight,keepaspectratio}
\IfFileExists{parskip.sty}{%
\usepackage{parskip}
}{% else
\setlength{\parindent}{0pt}
\setlength{\parskip}{6pt plus 2pt minus 1pt}
}
\setlength{\emergencystretch}{3em}  % prevent overfull lines
\providecommand{\tightlist}{%
  \setlength{\itemsep}{0pt}\setlength{\parskip}{0pt}}
\setcounter{secnumdepth}{0}
% Redefines (sub)paragraphs to behave more like sections
\ifx\paragraph\undefined\else
\let\oldparagraph\paragraph
\renewcommand{\paragraph}[1]{\oldparagraph{#1}\mbox{}}
\fi
\ifx\subparagraph\undefined\else
\let\oldsubparagraph\subparagraph
\renewcommand{\subparagraph}[1]{\oldsubparagraph{#1}\mbox{}}
\fi

%%% Use protect on footnotes to avoid problems with footnotes in titles
\let\rmarkdownfootnote\footnote%
\def\footnote{\protect\rmarkdownfootnote}

%%% Change title format to be more compact
\usepackage{titling}

% Create subtitle command for use in maketitle
\providecommand{\subtitle}[1]{
  \posttitle{
    \begin{center}\large#1\end{center}
    }
}

\setlength{\droptitle}{-2em}

  \title{Assign\_1(Vectors and Matrices)}
    \pretitle{\vspace{\droptitle}\centering\huge}
  \posttitle{\par}
    \author{Salma Elshahawy}
    \preauthor{\centering\large\emph}
  \postauthor{\par}
      \predate{\centering\large\emph}
  \postdate{\par}
    \date{1/29/2020}

\usepackage{geometry}

\begin{document}
\maketitle

\begin{Shaded}
\begin{Highlighting}[]
\ControlFlowTok{if}\NormalTok{ (}\OperatorTok{!}\KeywordTok{require}\NormalTok{(}\StringTok{"pacman"}\NormalTok{)) }\KeywordTok{install.packages}\NormalTok{(}\StringTok{"pacman"}\NormalTok{)}
\NormalTok{pacman}\OperatorTok{::}\KeywordTok{p_load}\NormalTok{(knitr, rmdformats, REdaS)}
\end{Highlighting}
\end{Shaded}

\hypertarget{problem-set-1}{%
\section{Problem set-1}\label{problem-set-1}}

You can think of vectors representing many dimensions of related
information. For instance, Netflix might store all the ratings a user
gives to movies in a vector. This is clearly a vector of very large
dimensions (in the millions) and very sparse as the user might have
rated only a few movies. Similarly, Amazon might store the items
purchased by a user in a vector, with each slot or dimension
representing a unique product and the value of the slot, the number of
such items the user bought. One task that is frequently done in these
settings is to find similarities between users. And, we can use
dot-product between vectors to do just that. As you know, the
dot-product is proportional to the length of two vectors and to the
angle between them. In fact, the dot-product between two vectors,
normalized by their lengths is called as the cosine distance and is
frequently used in recommendation engines.

\hypertarget{calculate-the-dot-product-u.v-where-u-0.5-0.5-and-v-3--4}{%
\subsubsection{1. Calculate the dot product u.v where u = {[}0.5, 0.5{]}
and v = {[}3,
-4{]}}\label{calculate-the-dot-product-u.v-where-u-0.5-0.5-and-v-3--4}}

\begin{Shaded}
\begin{Highlighting}[]
\NormalTok{u <-}\StringTok{ }\KeywordTok{c}\NormalTok{(}\FloatTok{0.5}\NormalTok{, }\FloatTok{0.5}\NormalTok{)}
\NormalTok{v <-}\StringTok{ }\KeywordTok{c}\NormalTok{(}\DecValTok{3}\NormalTok{, }\DecValTok{-4}\NormalTok{)}
\NormalTok{dot_prod =}\StringTok{ }\NormalTok{u }\OperatorTok\StringTok{ }\NormalTok{v}
\NormalTok{dot_prod}
\end{Highlighting}
\end{Shaded}

\begin{verbatim}
     [,1]
[1,] -0.5
\end{verbatim}

\hypertarget{what-are-the-lengths-of-u-and-v-please-note-that-the-mathematical-notion-of-the}{%
\subsubsection{2. What are the lengths of u and v? Please note that the
mathematical notion of
the}\label{what-are-the-lengths-of-u-and-v-please-note-that-the-mathematical-notion-of-the}}

length of a vector is not the same as a computer science definition.

\begin{Shaded}
\begin{Highlighting}[]
\NormalTok{v_length <-}\StringTok{ }\KeywordTok{sqrt}\NormalTok{((v[}\DecValTok{1}\NormalTok{]) }\OperatorTok{*}\StringTok{ }\NormalTok{(v[}\DecValTok{1}\NormalTok{]) }\OperatorTok{+}\StringTok{ }\NormalTok{(v[}\DecValTok{2}\NormalTok{]) }\OperatorTok{*}\StringTok{ }\NormalTok{(v[}\DecValTok{2}\NormalTok{]))}
\NormalTok{v_length}
\end{Highlighting}
\end{Shaded}

\begin{verbatim}
[1] 5
\end{verbatim}

\begin{Shaded}
\begin{Highlighting}[]
\NormalTok{u_length <-}\StringTok{ }\KeywordTok{sqrt}\NormalTok{((u[}\DecValTok{1}\NormalTok{]) }\OperatorTok{*}\StringTok{ }\NormalTok{(u[}\DecValTok{1}\NormalTok{]) }\OperatorTok{+}\StringTok{ }\NormalTok{(u[}\DecValTok{2}\NormalTok{]) }\OperatorTok{*}\StringTok{ }\NormalTok{(u[}\DecValTok{2}\NormalTok{]))}
\NormalTok{u_length}
\end{Highlighting}
\end{Shaded}

\begin{verbatim}
[1] 0.7071068
\end{verbatim}

\hypertarget{what-is-the-linear-combination-3u---2v}{%
\subsubsection{3. What is the linear combination: 3u -
2v?}\label{what-is-the-linear-combination-3u---2v}}

\begin{Shaded}
\begin{Highlighting}[]
\NormalTok{linear_comb <-}\StringTok{ }\NormalTok{(}\DecValTok{3} \OperatorTok{*}\StringTok{ }\NormalTok{u) }\OperatorTok{-}\StringTok{ }\NormalTok{(}\DecValTok{2} \OperatorTok{*}\StringTok{ }\NormalTok{v)}
\NormalTok{linear_comb}
\end{Highlighting}
\end{Shaded}

\begin{verbatim}
[1] -4.5  9.5
\end{verbatim}

\hypertarget{what-is-the-angle-between-u-and-v}{%
\subsubsection{4. What is the angle between u and
v?}\label{what-is-the-angle-between-u-and-v}}

\begin{Shaded}
\begin{Highlighting}[]
\NormalTok{cos_theta <-}\StringTok{ }\NormalTok{dot_prod}\OperatorTok{/}\NormalTok{(u_length }\OperatorTok{*}\StringTok{ }\NormalTok{v_length)}
\KeywordTok{rad2deg}\NormalTok{(}\KeywordTok{acos}\NormalTok{(cos_theta))}
\end{Highlighting}
\end{Shaded}

\begin{verbatim}
        [,1]
[1,] 98.1301
\end{verbatim}

\hypertarget{problem-set-2}{%
\section{Problem set-2}\label{problem-set-2}}

Set up a system of equations with 3 variables and 3 constraints and
solve for x. Please write a function in R that will take two variables
(matrix A \& constraint vector b) and solve using elimination. Your
function should produce the right answer for the system of equations for
any 3-variable, 3-equation system. You don't have to worry about
degenerate cases and can safely assume that the function will only be
tested with a system of equations that has a solution. Please note that
you do have to worry about zero pivots, though. Please note that you
should not use the built-in function solve to solve this system or use
matrix inverses. The approach that you should employ is to construct an
Upper Triangular Matrix and then back-substitute to get the solution.
Alternatively, you can augment the matrix A with vector b and jointly
apply the Gauss Jordan elimination procedure.

Please test it with the system below and it should produce a solution x
= {[}−1.55, −0.32, 0.95{]}

\[\begin{bmatrix} 1 & 1 & 3 \\ 2 & -1 & 5 \\ -1 & -2 & 4 \end{bmatrix}\quad \cdot \quad \begin{bmatrix} { x }_{ 1 } \\ { x }_{ 2 } \\ { x }_{ 3 } \end{bmatrix}\quad =\quad \begin{bmatrix} 1 \\ 2 \\ 6 \end{bmatrix}\]

\begin{Shaded}
\begin{Highlighting}[]
\NormalTok{calculate_gauss <-}\StringTok{ }\ControlFlowTok{function}\NormalTok{(a, b) \{}
\NormalTok{    n <-}\StringTok{ }\KeywordTok{nrow}\NormalTok{(a)}
    \CommentTok{# search for the position of the maximum element in the matrix for each}
    \CommentTok{# column diagonally.}
    \ControlFlowTok{for}\NormalTok{ (i }\ControlFlowTok{in} \KeywordTok{seq_len}\NormalTok{(n }\OperatorTok{-}\StringTok{ }\DecValTok{1}\NormalTok{)) \{}
\NormalTok{        j <-}\StringTok{ }\KeywordTok{which.max}\NormalTok{(a[i}\OperatorTok{:}\NormalTok{n, i]) }\OperatorTok{+}\StringTok{ }\NormalTok{i }\OperatorTok{-}\StringTok{ }\DecValTok{1}  \CommentTok{# pointer to the new index of max. element in a column}
        \ControlFlowTok{if}\NormalTok{ (j }\OperatorTok{!=}\StringTok{ }\NormalTok{i) \{}
\NormalTok{            a[}\KeywordTok{c}\NormalTok{(i, j), i}\OperatorTok{:}\NormalTok{n] <-}\StringTok{ }\NormalTok{a[}\KeywordTok{c}\NormalTok{(j, i), i}\OperatorTok{:}\NormalTok{n]}
            \CommentTok{# swap the two elements}
\NormalTok{            b[}\KeywordTok{c}\NormalTok{(i, j), ] <-}\StringTok{ }\NormalTok{b[}\KeywordTok{c}\NormalTok{(j, i), ]}
\NormalTok{        \}}
        \CommentTok{# find the multiplier to eliminate - iterate over the submatrix}
\NormalTok{        k <-}\StringTok{ }\KeywordTok{seq}\NormalTok{(i }\OperatorTok{+}\StringTok{ }\DecValTok{1}\NormalTok{, n)}
        \ControlFlowTok{for}\NormalTok{ (j }\ControlFlowTok{in}\NormalTok{ k) \{}
            \CommentTok{# find the multiplier}
\NormalTok{            s <-}\StringTok{ }\NormalTok{a[[j, i]]}\OperatorTok{/}\NormalTok{a[[i, i]]}
            \CommentTok{# multiply s to the 2nd row then subtract from the first -> substitute in}
            \CommentTok{# the second}
\NormalTok{            a[j, k] <-}\StringTok{ }\NormalTok{a[j, k] }\OperatorTok{-}\StringTok{ }\NormalTok{s }\OperatorTok{*}\StringTok{ }\NormalTok{a[i, k]}
\NormalTok{            b[j, ] <-}\StringTok{ }\NormalTok{b[j, ] }\OperatorTok{-}\StringTok{ }\NormalTok{s }\OperatorTok{*}\StringTok{ }\NormalTok{b[i, ]}
\NormalTok{        \}}
\NormalTok{    \}}
    
    \CommentTok{# backword solve}
    \ControlFlowTok{for}\NormalTok{ (i }\ControlFlowTok{in} \KeywordTok{seq}\NormalTok{(n, }\DecValTok{1}\NormalTok{)) \{}
        \ControlFlowTok{if}\NormalTok{ (i }\OperatorTok{<}\StringTok{ }\NormalTok{n) \{}
            \ControlFlowTok{for}\NormalTok{ (j }\ControlFlowTok{in} \KeywordTok{seq}\NormalTok{(i }\OperatorTok{+}\StringTok{ }\DecValTok{1}\NormalTok{, n)) \{}
\NormalTok{                b[i, ] <-}\StringTok{ }\NormalTok{b[i, ] }\OperatorTok{-}\StringTok{ }\NormalTok{a[[i, j]] }\OperatorTok{*}\StringTok{ }\NormalTok{b[j, ]}
\NormalTok{            \}}
\NormalTok{        \}}
\NormalTok{        b[i, ] <-}\StringTok{ }\NormalTok{b[i, ]}\OperatorTok{/}\NormalTok{a[[i, i]]}
\NormalTok{    \}}
    
    \KeywordTok{return}\NormalTok{(}\DataTypeTok{x =}\NormalTok{ b)}
\NormalTok{\}}
\NormalTok{a <-}\StringTok{ }\KeywordTok{matrix}\NormalTok{(}\KeywordTok{c}\NormalTok{(}\DecValTok{1}\NormalTok{, }\DecValTok{2}\NormalTok{, }\DecValTok{-1}\NormalTok{, }\DecValTok{1}\NormalTok{, }\DecValTok{-1}\NormalTok{, }\DecValTok{-2}\NormalTok{, }\DecValTok{3}\NormalTok{, }\DecValTok{5}\NormalTok{, }\DecValTok{4}\NormalTok{), }\DataTypeTok{nrow =} \DecValTok{3}\NormalTok{, }\DataTypeTok{ncol =} \DecValTok{3}\NormalTok{)}
\NormalTok{b <-}\StringTok{ }\KeywordTok{matrix}\NormalTok{(}\KeywordTok{c}\NormalTok{(}\DecValTok{1}\NormalTok{, }\DecValTok{2}\NormalTok{, }\DecValTok{6}\NormalTok{), }\DataTypeTok{nrow =} \DecValTok{3}\NormalTok{, }\DataTypeTok{ncol =} \DecValTok{1}\NormalTok{)}

\NormalTok{val <-}\StringTok{ }\KeywordTok{calculate_gauss}\NormalTok{(a, b)}
\NormalTok{val}
\end{Highlighting}
\end{Shaded}

\begin{verbatim}
           [,1]
[1,] -1.5454545
[2,] -0.3181818
[3,]  0.9545455
\end{verbatim}


\end{document}
