\documentclass[]{article}
\usepackage{lmodern}
\usepackage{amssymb,amsmath}
\usepackage{ifxetex,ifluatex}
\usepackage{fixltx2e} % provides \textsubscript
\ifnum 0\ifxetex 1\fi\ifluatex 1\fi=0 % if pdftex
  \usepackage[T1]{fontenc}
  \usepackage[utf8]{inputenc}
\else % if luatex or xelatex
  \ifxetex
    \usepackage{mathspec}
  \else
    \usepackage{fontspec}
  \fi
  \defaultfontfeatures{Ligatures=TeX,Scale=MatchLowercase}
\fi
% use upquote if available, for straight quotes in verbatim environments
\IfFileExists{upquote.sty}{\usepackage{upquote}}{}
% use microtype if available
\IfFileExists{microtype.sty}{%
\usepackage{microtype}
\UseMicrotypeSet[protrusion]{basicmath} % disable protrusion for tt fonts
}{}
\usepackage[margin=1in]{geometry}
\usepackage{hyperref}
\hypersetup{unicode=true,
            pdftitle={Week\_3 605 assignment},
            pdfauthor={Salma Elshahawy},
            pdfborder={0 0 0},
            breaklinks=true}
\urlstyle{same}  % don't use monospace font for urls
\usepackage{color}
\usepackage{fancyvrb}
\newcommand{\VerbBar}{|}
\newcommand{\VERB}{\Verb[commandchars=\\\{\}]}
\DefineVerbatimEnvironment{Highlighting}{Verbatim}{commandchars=\\\{\}}
% Add ',fontsize=\small' for more characters per line
\usepackage{framed}
\definecolor{shadecolor}{RGB}{248,248,248}
\newenvironment{Shaded}{\begin{snugshade}}{\end{snugshade}}
\newcommand{\AlertTok}[1]{\textcolor[rgb]{0.94,0.16,0.16}{#1}}
\newcommand{\AnnotationTok}[1]{\textcolor[rgb]{0.56,0.35,0.01}{\textbf{\textit{#1}}}}
\newcommand{\AttributeTok}[1]{\textcolor[rgb]{0.77,0.63,0.00}{#1}}
\newcommand{\BaseNTok}[1]{\textcolor[rgb]{0.00,0.00,0.81}{#1}}
\newcommand{\BuiltInTok}[1]{#1}
\newcommand{\CharTok}[1]{\textcolor[rgb]{0.31,0.60,0.02}{#1}}
\newcommand{\CommentTok}[1]{\textcolor[rgb]{0.56,0.35,0.01}{\textit{#1}}}
\newcommand{\CommentVarTok}[1]{\textcolor[rgb]{0.56,0.35,0.01}{\textbf{\textit{#1}}}}
\newcommand{\ConstantTok}[1]{\textcolor[rgb]{0.00,0.00,0.00}{#1}}
\newcommand{\ControlFlowTok}[1]{\textcolor[rgb]{0.13,0.29,0.53}{\textbf{#1}}}
\newcommand{\DataTypeTok}[1]{\textcolor[rgb]{0.13,0.29,0.53}{#1}}
\newcommand{\DecValTok}[1]{\textcolor[rgb]{0.00,0.00,0.81}{#1}}
\newcommand{\DocumentationTok}[1]{\textcolor[rgb]{0.56,0.35,0.01}{\textbf{\textit{#1}}}}
\newcommand{\ErrorTok}[1]{\textcolor[rgb]{0.64,0.00,0.00}{\textbf{#1}}}
\newcommand{\ExtensionTok}[1]{#1}
\newcommand{\FloatTok}[1]{\textcolor[rgb]{0.00,0.00,0.81}{#1}}
\newcommand{\FunctionTok}[1]{\textcolor[rgb]{0.00,0.00,0.00}{#1}}
\newcommand{\ImportTok}[1]{#1}
\newcommand{\InformationTok}[1]{\textcolor[rgb]{0.56,0.35,0.01}{\textbf{\textit{#1}}}}
\newcommand{\KeywordTok}[1]{\textcolor[rgb]{0.13,0.29,0.53}{\textbf{#1}}}
\newcommand{\NormalTok}[1]{#1}
\newcommand{\OperatorTok}[1]{\textcolor[rgb]{0.81,0.36,0.00}{\textbf{#1}}}
\newcommand{\OtherTok}[1]{\textcolor[rgb]{0.56,0.35,0.01}{#1}}
\newcommand{\PreprocessorTok}[1]{\textcolor[rgb]{0.56,0.35,0.01}{\textit{#1}}}
\newcommand{\RegionMarkerTok}[1]{#1}
\newcommand{\SpecialCharTok}[1]{\textcolor[rgb]{0.00,0.00,0.00}{#1}}
\newcommand{\SpecialStringTok}[1]{\textcolor[rgb]{0.31,0.60,0.02}{#1}}
\newcommand{\StringTok}[1]{\textcolor[rgb]{0.31,0.60,0.02}{#1}}
\newcommand{\VariableTok}[1]{\textcolor[rgb]{0.00,0.00,0.00}{#1}}
\newcommand{\VerbatimStringTok}[1]{\textcolor[rgb]{0.31,0.60,0.02}{#1}}
\newcommand{\WarningTok}[1]{\textcolor[rgb]{0.56,0.35,0.01}{\textbf{\textit{#1}}}}
\usepackage{graphicx,grffile}
\makeatletter
\def\maxwidth{\ifdim\Gin@nat@width>\linewidth\linewidth\else\Gin@nat@width\fi}
\def\maxheight{\ifdim\Gin@nat@height>\textheight\textheight\else\Gin@nat@height\fi}
\makeatother
% Scale images if necessary, so that they will not overflow the page
% margins by default, and it is still possible to overwrite the defaults
% using explicit options in \includegraphics[width, height, ...]{}
\setkeys{Gin}{width=\maxwidth,height=\maxheight,keepaspectratio}
\IfFileExists{parskip.sty}{%
\usepackage{parskip}
}{% else
\setlength{\parindent}{0pt}
\setlength{\parskip}{6pt plus 2pt minus 1pt}
}
\setlength{\emergencystretch}{3em}  % prevent overfull lines
\providecommand{\tightlist}{%
  \setlength{\itemsep}{0pt}\setlength{\parskip}{0pt}}
\setcounter{secnumdepth}{0}
% Redefines (sub)paragraphs to behave more like sections
\ifx\paragraph\undefined\else
\let\oldparagraph\paragraph
\renewcommand{\paragraph}[1]{\oldparagraph{#1}\mbox{}}
\fi
\ifx\subparagraph\undefined\else
\let\oldsubparagraph\subparagraph
\renewcommand{\subparagraph}[1]{\oldsubparagraph{#1}\mbox{}}
\fi

%%% Use protect on footnotes to avoid problems with footnotes in titles
\let\rmarkdownfootnote\footnote%
\def\footnote{\protect\rmarkdownfootnote}

%%% Change title format to be more compact
\usepackage{titling}

% Create subtitle command for use in maketitle
\providecommand{\subtitle}[1]{
  \posttitle{
    \begin{center}\large#1\end{center}
    }
}

\setlength{\droptitle}{-2em}

  \title{Week\_3 605 assignment}
    \pretitle{\vspace{\droptitle}\centering\huge}
  \posttitle{\par}
    \author{Salma Elshahawy}
    \preauthor{\centering\large\emph}
  \postauthor{\par}
      \predate{\centering\large\emph}
  \postdate{\par}
    \date{1/21/2020}

\usepackage{geometry}

\begin{document}
\maketitle

\hypertarget{problem-set_1}{%
\subsection{Problem set\_1}\label{problem-set_1}}

\hypertarget{what-is-the-rank-of-the-matrix-a}{%
\subsubsection{(1) What is the rank of the matrix
A?}\label{what-is-the-rank-of-the-matrix-a}}

\(\begin{bmatrix} 1 & 2 & 3 & 4 \\ -1 & 0 & 1 & 3 \\ 0 & 1 & -2 & 1 \\ 5 & 4 & -2 & -3 \end{bmatrix}\)

\begin{Shaded}
\begin{Highlighting}[]
\NormalTok{a_mEn <-}\StringTok{ }\KeywordTok{matrix}\NormalTok{(}\KeywordTok{c}\NormalTok{(}\DecValTok{1}\NormalTok{,}\DecValTok{2}\NormalTok{,}\DecValTok{3}\NormalTok{,}\DecValTok{4}\NormalTok{,}
              \DecValTok{-1}\NormalTok{,}\DecValTok{0}\NormalTok{,}\DecValTok{1}\NormalTok{,}\DecValTok{3}\NormalTok{,}
              \DecValTok{0}\NormalTok{,}\DecValTok{1}\NormalTok{,}\OperatorTok{-}\DecValTok{2}\NormalTok{,}\DecValTok{1}\NormalTok{,}
              \DecValTok{5}\NormalTok{,}\DecValTok{4}\NormalTok{,}\OperatorTok{-}\DecValTok{2}\NormalTok{,}\OperatorTok{-}\DecValTok{3}\NormalTok{), }\DecValTok{4}\NormalTok{, }\DataTypeTok{byrow=}\NormalTok{T)}
\CommentTok{# a <- matrix(c(0,1,2,1,2,7,2,1,8), ncol = 3)}
\NormalTok{a_mEn}
\end{Highlighting}
\end{Shaded}

\begin{verbatim}
##      [,1] [,2] [,3] [,4]
## [1,]    1    2    3    4
## [2,]   -1    0    1    3
## [3,]    0    1   -2    1
## [4,]    5    4   -2   -3
\end{verbatim}

\begin{Shaded}
\begin{Highlighting}[]
\NormalTok{a_mbn <-}\StringTok{ }\KeywordTok{matrix}\NormalTok{(}\KeywordTok{c}\NormalTok{(}\DecValTok{1}\NormalTok{,}\DecValTok{2}\NormalTok{,}\DecValTok{3}\NormalTok{,}\DecValTok{4}\NormalTok{,}
              \DecValTok{-1}\NormalTok{,}\DecValTok{0}\NormalTok{,}\DecValTok{1}\NormalTok{,}\DecValTok{3}\NormalTok{), }\DecValTok{4}\NormalTok{, }\DataTypeTok{byrow=}\NormalTok{T)}
\NormalTok{a_mbn}
\end{Highlighting}
\end{Shaded}

\begin{verbatim}
##      [,1] [,2]
## [1,]    1    2
## [2,]    3    4
## [3,]   -1    0
## [4,]    1    3
\end{verbatim}

\begin{Shaded}
\begin{Highlighting}[]
\NormalTok{a_mln <-}\StringTok{ }\KeywordTok{matrix}\NormalTok{(}\KeywordTok{c}\NormalTok{(}\DecValTok{1}\NormalTok{,}\DecValTok{2}\NormalTok{,}\DecValTok{3}\NormalTok{,}\DecValTok{4}\NormalTok{,}
              \DecValTok{-1}\NormalTok{,}\DecValTok{0}\NormalTok{,}\DecValTok{1}\NormalTok{,}\DecValTok{3}\NormalTok{), }\DecValTok{2}\NormalTok{, }\DataTypeTok{byrow=}\NormalTok{T)}
\NormalTok{a_mln}
\end{Highlighting}
\end{Shaded}

\begin{verbatim}
##      [,1] [,2] [,3] [,4]
## [1,]    1    2    3    4
## [2,]   -1    0    1    3
\end{verbatim}

\begin{Shaded}
\begin{Highlighting}[]
\NormalTok{get_echoln <-}\StringTok{ }\ControlFlowTok{function}\NormalTok{(a) \{}
\NormalTok{  U =}\StringTok{ }\NormalTok{a}
\NormalTok{  n =}\StringTok{ }\KeywordTok{ncol}\NormalTok{(a)}
\NormalTok{  m =}\StringTok{ }\KeywordTok{nrow}\NormalTok{(a)}
  \ControlFlowTok{if}\NormalTok{(m }\OperatorTok{==}\StringTok{ }\NormalTok{n) \{}
    \ControlFlowTok{for}\NormalTok{ (i }\ControlFlowTok{in} \DecValTok{1}\OperatorTok{:}\NormalTok{n) \{}
      \ControlFlowTok{for}\NormalTok{ (j }\ControlFlowTok{in} \DecValTok{2}\OperatorTok{:}\NormalTok{m) \{}
        \ControlFlowTok{if}\NormalTok{(U[j,i] }\OperatorTok{!=}\StringTok{ }\DecValTok{0} \OperatorTok{&}\StringTok{ }\NormalTok{j }\OperatorTok{>}\StringTok{ }\NormalTok{i) \{}
          \CommentTok{# Add multiples of the pivot row to each of the lower rows, }
          \CommentTok{# so every element in the pivot column of the lower rows equals 0.}
\NormalTok{          mplier =}\StringTok{ }\NormalTok{U[[j,i]]}\OperatorTok{/}\NormalTok{U[[i,i]]}
          \CommentTok{# reduce by reduction and subtitute in the U matrix}
\NormalTok{          U[j,] =}\StringTok{ }\NormalTok{U[j,] }\OperatorTok{-}\StringTok{ }\NormalTok{mplier }\OperatorTok{*}\StringTok{ }\NormalTok{U[i,]}
\NormalTok{        \} }\ControlFlowTok{else} \ControlFlowTok{if}\NormalTok{ (j }\OperatorTok{==}\StringTok{ }\NormalTok{i) \{}
\NormalTok{          U[j,] =}\StringTok{ }\NormalTok{U[j,] }\OperatorTok{/}\StringTok{ }\NormalTok{U[[j,i]]}
\NormalTok{        \}}\CommentTok{# end if}
\NormalTok{      \} }\CommentTok{# end if }
\NormalTok{    \} }\CommentTok{# end for}
\NormalTok{  \} }\ControlFlowTok{else} \ControlFlowTok{if}\NormalTok{(m }\OperatorTok{<}\StringTok{ }\NormalTok{n) \{}
    \ControlFlowTok{for}\NormalTok{ (i }\ControlFlowTok{in} \DecValTok{1}\OperatorTok{:}\NormalTok{n) \{}
      \ControlFlowTok{for}\NormalTok{ (j }\ControlFlowTok{in} \DecValTok{2}\OperatorTok{:}\NormalTok{m) \{}
        \ControlFlowTok{if}\NormalTok{(U[j,i] }\OperatorTok{!=}\StringTok{ }\DecValTok{0} \OperatorTok{&}\StringTok{ }\NormalTok{j }\OperatorTok{>}\StringTok{ }\NormalTok{i) \{}
\NormalTok{          U[i,] =}\StringTok{ }\NormalTok{U[i,] }\OperatorTok{/}\StringTok{ }\NormalTok{U[[i,i]]}
          \CommentTok{# Add multiples of the pivot row to each of the lower rows, }
          \CommentTok{# so every element in the pivot column of the lower rows equals 0.}
\NormalTok{          mplier =}\StringTok{ }\NormalTok{U[[j,i]]}\OperatorTok{/}\NormalTok{U[[i,i]]}
          \CommentTok{# reduce by reduction and subtitute in the U matrix}
\NormalTok{          U[j,] =}\StringTok{ }\NormalTok{U[j,] }\OperatorTok{-}\StringTok{ }\NormalTok{mplier }\OperatorTok{*}\StringTok{ }\NormalTok{U[i,]}
          
\NormalTok{        \} }\ControlFlowTok{else} \ControlFlowTok{if}\NormalTok{(U[j,i] }\OperatorTok{!=}\StringTok{ }\DecValTok{0} \OperatorTok{&}\StringTok{ }\NormalTok{j }\OperatorTok{==}\StringTok{ }\NormalTok{i) \{}
\NormalTok{          U[i,] =}\StringTok{ }\NormalTok{U[i,] }\OperatorTok{/}\StringTok{ }\NormalTok{U[[i,i]]}
\NormalTok{        \} }\CommentTok{# end if}
\NormalTok{      \} }\CommentTok{# end for}
\NormalTok{    \} }\CommentTok{# end for}
\NormalTok{  \} }\ControlFlowTok{else} \ControlFlowTok{if}\NormalTok{ (m }\OperatorTok{>}\StringTok{ }\NormalTok{n) \{}
    \ControlFlowTok{for}\NormalTok{ (i }\ControlFlowTok{in} \DecValTok{1}\OperatorTok{:}\NormalTok{n) \{}
      \ControlFlowTok{for}\NormalTok{ (j }\ControlFlowTok{in} \DecValTok{2}\OperatorTok{:}\NormalTok{m) \{}
        \ControlFlowTok{if}\NormalTok{(U[j,i] }\OperatorTok{!=}\StringTok{ }\DecValTok{0} \OperatorTok{&}\StringTok{ }\NormalTok{j }\OperatorTok{>}\StringTok{ }\NormalTok{i) \{}
\NormalTok{          U[i,] =}\StringTok{ }\NormalTok{U[i,] }\OperatorTok{/}\StringTok{ }\NormalTok{U[[i,i]]}
          \CommentTok{# Add multiples of the pivot row to each of the lower rows, }
          \CommentTok{# so every element in the pivot column of the lower rows equals 0.}
\NormalTok{          mplier =}\StringTok{ }\NormalTok{U[[j,i]]}\OperatorTok{/}\NormalTok{U[[i,i]]}
          \CommentTok{# reduce by reduction and subtitute in the U matrix}
\NormalTok{          U[j,] =}\StringTok{ }\NormalTok{U[j,] }\OperatorTok{-}\StringTok{ }\NormalTok{mplier }\OperatorTok{*}\StringTok{ }\NormalTok{U[i,]}
          
\NormalTok{        \} }\ControlFlowTok{else} \ControlFlowTok{if}\NormalTok{(U[j,i] }\OperatorTok{!=}\StringTok{ }\DecValTok{0} \OperatorTok{&}\StringTok{ }\NormalTok{j }\OperatorTok{==}\StringTok{ }\NormalTok{i) \{}
\NormalTok{          U[i,] =}\StringTok{ }\NormalTok{U[i,] }\OperatorTok{/}\StringTok{ }\NormalTok{U[[i,i]]}
\NormalTok{        \} }\CommentTok{# end if}
\NormalTok{      \} }\CommentTok{# end for}
\NormalTok{    \} }\CommentTok{# end for}
\NormalTok{  \} }\CommentTok{# end if }
  \KeywordTok{return}\NormalTok{(}\KeywordTok{round}\NormalTok{(U, }\DataTypeTok{digits =} \DecValTok{1}\NormalTok{))}
\NormalTok{\}}

\NormalTok{equal =}\StringTok{ }\KeywordTok{get_echoln}\NormalTok{(a_mEn)}
\NormalTok{equal}
\end{Highlighting}
\end{Shaded}

\begin{verbatim}
##      [,1] [,2] [,3] [,4]
## [1,]    1    2    3  4.0
## [2,]    0    1    2  3.5
## [3,]    0    0    1  0.6
## [4,]    0    0    0  1.0
\end{verbatim}

\begin{Shaded}
\begin{Highlighting}[]
\NormalTok{greater =}\StringTok{ }\KeywordTok{get_echoln}\NormalTok{(a_mbn)}
\NormalTok{greater}
\end{Highlighting}
\end{Shaded}

\begin{verbatim}
##      [,1] [,2]
## [1,]    1    2
## [2,]    0    1
## [3,]    0    0
## [4,]    0    0
\end{verbatim}

\begin{Shaded}
\begin{Highlighting}[]
\NormalTok{lesser =}\StringTok{ }\KeywordTok{get_echoln}\NormalTok{(a_mln)}
\NormalTok{lesser}
\end{Highlighting}
\end{Shaded}

\begin{verbatim}
##      [,1] [,2] [,3] [,4]
## [1,]    1    2    3  4.0
## [2,]    0    1    2  3.5
\end{verbatim}

\begin{Shaded}
\begin{Highlighting}[]
\CommentTok{# rank needs to be modified}
\NormalTok{ranking =}\StringTok{ }\ControlFlowTok{function}\NormalTok{(cd) \{}
\NormalTok{  rank =}\StringTok{ }\DecValTok{0}
\NormalTok{  sol =}\StringTok{ }\KeywordTok{as.array}\NormalTok{(}\KeywordTok{colSums}\NormalTok{(cd))}
  \CommentTok{# [1]  1  3  6 10}
  \ControlFlowTok{for}\NormalTok{ (i }\ControlFlowTok{in} \DecValTok{1}\OperatorTok{:}\KeywordTok{length}\NormalTok{(sol)) \{}
    \ControlFlowTok{if}\NormalTok{(sol[i] }\OperatorTok{!=}\StringTok{ }\DecValTok{0} \OperatorTok{&}\StringTok{ }\KeywordTok{ncol}\NormalTok{(cd) }\OperatorTok{<=}\StringTok{ }\KeywordTok{nrow}\NormalTok{(cd)) \{}
\NormalTok{      rank =}\StringTok{ }\NormalTok{rank }\OperatorTok{+}\StringTok{ }\DecValTok{1}
\NormalTok{    \} }\ControlFlowTok{else} \ControlFlowTok{if}\NormalTok{ (sol[i] }\OperatorTok{!=}\StringTok{ }\DecValTok{0} \OperatorTok{&}\StringTok{ }\KeywordTok{ncol}\NormalTok{(cd) }\OperatorTok{>}\StringTok{ }\KeywordTok{nrow}\NormalTok{(cd)) \{}
\NormalTok{      rank =}\StringTok{ }\KeywordTok{nrow}\NormalTok{(cd)}
\NormalTok{    \}}
\NormalTok{  \}}
  \KeywordTok{return}\NormalTok{(rank)}
\NormalTok{\}}

\NormalTok{r1 =}\StringTok{ }\KeywordTok{ranking}\NormalTok{(equal)}
\NormalTok{r1}
\end{Highlighting}
\end{Shaded}

\begin{verbatim}
## [1] 4
\end{verbatim}

\begin{Shaded}
\begin{Highlighting}[]
\NormalTok{r2 =}\StringTok{ }\KeywordTok{ranking}\NormalTok{(greater)}
\NormalTok{r2}
\end{Highlighting}
\end{Shaded}

\begin{verbatim}
## [1] 2
\end{verbatim}

\begin{Shaded}
\begin{Highlighting}[]
\NormalTok{r3 =}\StringTok{ }\KeywordTok{ranking}\NormalTok{(lesser)}
\NormalTok{r3}
\end{Highlighting}
\end{Shaded}

\begin{verbatim}
## [1] 2
\end{verbatim}

\hypertarget{given-an-mtimes-matrix-where-m-n-what-can-be-the-maximum-rank-the-mini--mum-rank-assuming-that-the-matrix-is-non-zero}{%
\subsubsection{\texorpdfstring{(2) Given an \(m\times\) matrix where m
\textgreater{} n, what can be the maximum rank? The mini- mum rank,
assuming that the matrix is
non-zero?}{(2) Given an m\textbackslash times matrix where m \textgreater{} n, what can be the maximum rank? The mini- mum rank, assuming that the matrix is non-zero?}}\label{given-an-mtimes-matrix-where-m-n-what-can-be-the-maximum-rank-the-mini--mum-rank-assuming-that-the-matrix-is-non-zero}}


\end{document}
