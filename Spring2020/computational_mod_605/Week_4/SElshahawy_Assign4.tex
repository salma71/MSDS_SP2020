\documentclass[]{article}
\usepackage{lmodern}
\usepackage{amssymb,amsmath}
\usepackage{ifxetex,ifluatex}
\usepackage{fixltx2e} % provides \textsubscript
\ifnum 0\ifxetex 1\fi\ifluatex 1\fi=0 % if pdftex
  \usepackage[T1]{fontenc}
  \usepackage[utf8]{inputenc}
\else % if luatex or xelatex
  \ifxetex
    \usepackage{mathspec}
  \else
    \usepackage{fontspec}
  \fi
  \defaultfontfeatures{Ligatures=TeX,Scale=MatchLowercase}
\fi
% use upquote if available, for straight quotes in verbatim environments
\IfFileExists{upquote.sty}{\usepackage{upquote}}{}
% use microtype if available
\IfFileExists{microtype.sty}{%
\usepackage{microtype}
\UseMicrotypeSet[protrusion]{basicmath} % disable protrusion for tt fonts
}{}
\usepackage[margin=1in]{geometry}
\usepackage{hyperref}
\hypersetup{unicode=true,
            pdftitle={SElshahawy\_Assign4},
            pdfauthor={Salma Elshahawy},
            pdfborder={0 0 0},
            breaklinks=true}
\urlstyle{same}  % don't use monospace font for urls
\usepackage{color}
\usepackage{fancyvrb}
\newcommand{\VerbBar}{|}
\newcommand{\VERB}{\Verb[commandchars=\\\{\}]}
\DefineVerbatimEnvironment{Highlighting}{Verbatim}{commandchars=\\\{\}}
% Add ',fontsize=\small' for more characters per line
\usepackage{framed}
\definecolor{shadecolor}{RGB}{248,248,248}
\newenvironment{Shaded}{\begin{snugshade}}{\end{snugshade}}
\newcommand{\AlertTok}[1]{\textcolor[rgb]{0.94,0.16,0.16}{#1}}
\newcommand{\AnnotationTok}[1]{\textcolor[rgb]{0.56,0.35,0.01}{\textbf{\textit{#1}}}}
\newcommand{\AttributeTok}[1]{\textcolor[rgb]{0.77,0.63,0.00}{#1}}
\newcommand{\BaseNTok}[1]{\textcolor[rgb]{0.00,0.00,0.81}{#1}}
\newcommand{\BuiltInTok}[1]{#1}
\newcommand{\CharTok}[1]{\textcolor[rgb]{0.31,0.60,0.02}{#1}}
\newcommand{\CommentTok}[1]{\textcolor[rgb]{0.56,0.35,0.01}{\textit{#1}}}
\newcommand{\CommentVarTok}[1]{\textcolor[rgb]{0.56,0.35,0.01}{\textbf{\textit{#1}}}}
\newcommand{\ConstantTok}[1]{\textcolor[rgb]{0.00,0.00,0.00}{#1}}
\newcommand{\ControlFlowTok}[1]{\textcolor[rgb]{0.13,0.29,0.53}{\textbf{#1}}}
\newcommand{\DataTypeTok}[1]{\textcolor[rgb]{0.13,0.29,0.53}{#1}}
\newcommand{\DecValTok}[1]{\textcolor[rgb]{0.00,0.00,0.81}{#1}}
\newcommand{\DocumentationTok}[1]{\textcolor[rgb]{0.56,0.35,0.01}{\textbf{\textit{#1}}}}
\newcommand{\ErrorTok}[1]{\textcolor[rgb]{0.64,0.00,0.00}{\textbf{#1}}}
\newcommand{\ExtensionTok}[1]{#1}
\newcommand{\FloatTok}[1]{\textcolor[rgb]{0.00,0.00,0.81}{#1}}
\newcommand{\FunctionTok}[1]{\textcolor[rgb]{0.00,0.00,0.00}{#1}}
\newcommand{\ImportTok}[1]{#1}
\newcommand{\InformationTok}[1]{\textcolor[rgb]{0.56,0.35,0.01}{\textbf{\textit{#1}}}}
\newcommand{\KeywordTok}[1]{\textcolor[rgb]{0.13,0.29,0.53}{\textbf{#1}}}
\newcommand{\NormalTok}[1]{#1}
\newcommand{\OperatorTok}[1]{\textcolor[rgb]{0.81,0.36,0.00}{\textbf{#1}}}
\newcommand{\OtherTok}[1]{\textcolor[rgb]{0.56,0.35,0.01}{#1}}
\newcommand{\PreprocessorTok}[1]{\textcolor[rgb]{0.56,0.35,0.01}{\textit{#1}}}
\newcommand{\RegionMarkerTok}[1]{#1}
\newcommand{\SpecialCharTok}[1]{\textcolor[rgb]{0.00,0.00,0.00}{#1}}
\newcommand{\SpecialStringTok}[1]{\textcolor[rgb]{0.31,0.60,0.02}{#1}}
\newcommand{\StringTok}[1]{\textcolor[rgb]{0.31,0.60,0.02}{#1}}
\newcommand{\VariableTok}[1]{\textcolor[rgb]{0.00,0.00,0.00}{#1}}
\newcommand{\VerbatimStringTok}[1]{\textcolor[rgb]{0.31,0.60,0.02}{#1}}
\newcommand{\WarningTok}[1]{\textcolor[rgb]{0.56,0.35,0.01}{\textbf{\textit{#1}}}}
\usepackage{graphicx,grffile}
\makeatletter
\def\maxwidth{\ifdim\Gin@nat@width>\linewidth\linewidth\else\Gin@nat@width\fi}
\def\maxheight{\ifdim\Gin@nat@height>\textheight\textheight\else\Gin@nat@height\fi}
\makeatother
% Scale images if necessary, so that they will not overflow the page
% margins by default, and it is still possible to overwrite the defaults
% using explicit options in \includegraphics[width, height, ...]{}
\setkeys{Gin}{width=\maxwidth,height=\maxheight,keepaspectratio}
\IfFileExists{parskip.sty}{%
\usepackage{parskip}
}{% else
\setlength{\parindent}{0pt}
\setlength{\parskip}{6pt plus 2pt minus 1pt}
}
\setlength{\emergencystretch}{3em}  % prevent overfull lines
\providecommand{\tightlist}{%
  \setlength{\itemsep}{0pt}\setlength{\parskip}{0pt}}
\setcounter{secnumdepth}{0}
% Redefines (sub)paragraphs to behave more like sections
\ifx\paragraph\undefined\else
\let\oldparagraph\paragraph
\renewcommand{\paragraph}[1]{\oldparagraph{#1}\mbox{}}
\fi
\ifx\subparagraph\undefined\else
\let\oldsubparagraph\subparagraph
\renewcommand{\subparagraph}[1]{\oldsubparagraph{#1}\mbox{}}
\fi

%%% Use protect on footnotes to avoid problems with footnotes in titles
\let\rmarkdownfootnote\footnote%
\def\footnote{\protect\rmarkdownfootnote}

%%% Change title format to be more compact
\usepackage{titling}

% Create subtitle command for use in maketitle
\providecommand{\subtitle}[1]{
  \posttitle{
    \begin{center}\large#1\end{center}
    }
}

\setlength{\droptitle}{-2em}

  \title{SElshahawy\_Assign4}
    \pretitle{\vspace{\droptitle}\centering\huge}
  \posttitle{\par}
    \author{Salma Elshahawy}
    \preauthor{\centering\large\emph}
  \postauthor{\par}
      \predate{\centering\large\emph}
  \postdate{\par}
    \date{2/5/2020}

\usepackage{geometry}

\begin{document}
\maketitle

\hypertarget{problem_set1}{%
\subsection{Problem\_set1}\label{problem_set1}}

In this problem, we'll verify using R that SVD and Eigenvalues are
related as worked out in the weekly module. Given a 3 x 2 matrix A write
code in R to 1. compute \(X = AA^T\) and \(Y = A^TA\). Then, 2. compute
the eigenvalues and eigenvectors of X and Y using the built-in commands
in R.

\[A=\begin{bmatrix}1&2&3\\-1&0&4\end{bmatrix}\]

Then, 3. compute the left-singular, singular values, and right-singular
vectors of A using the svd command. 4. Examine the two sets of singular
vectors and show that they are indeed eigenvectors of X and Y. In
addition, the two non-zero eigenvalues (the 3rd value will be very close
to zero, if not zero) of both X and Y are the same and are squares of
the non-zero singular values of A.

Your code should compute all these vectors and scalars and store them in
variables. Please add enough comments in your code to show me how to
interpret your steps.

\textbf{Answer:-}

\hypertarget{compute-x-and-y-using-built-in-functions}{%
\subsubsection{Compute X and Y using built-in
functions}\label{compute-x-and-y-using-built-in-functions}}

\begin{Shaded}
\begin{Highlighting}[]
\NormalTok{A <-}\StringTok{ }\KeywordTok{matrix}\NormalTok{(}\KeywordTok{c}\NormalTok{(}\DecValTok{1}\NormalTok{,}\DecValTok{2}\NormalTok{,}\DecValTok{3}\NormalTok{,}
              \DecValTok{-1}\NormalTok{,}\DecValTok{0}\NormalTok{,}\DecValTok{4}\NormalTok{), }\DataTypeTok{nrow =} \DecValTok{2}\NormalTok{, }\DataTypeTok{byrow=}\NormalTok{T)}
\NormalTok{A}
\end{Highlighting}
\end{Shaded}

\begin{verbatim}
##      [,1] [,2] [,3]
## [1,]    1    2    3
## [2,]   -1    0    4
\end{verbatim}

\begin{Shaded}
\begin{Highlighting}[]
\CommentTok{#  getting X value using t() function}
\NormalTok{X <-}\StringTok{ }\NormalTok{A }\OperatorTok\StringTok{ }\KeywordTok{t}\NormalTok{(A)}
\NormalTok{X}
\end{Highlighting}
\end{Shaded}

\begin{verbatim}
##      [,1] [,2]
## [1,]   14   11
## [2,]   11   17
\end{verbatim}

\begin{Shaded}
\begin{Highlighting}[]
\CommentTok{# doing the same for y}
\NormalTok{Y <-}\StringTok{ }\KeywordTok{t}\NormalTok{(A) }\OperatorTok\StringTok{ }\NormalTok{A}
\NormalTok{Y}
\end{Highlighting}
\end{Shaded}

\begin{verbatim}
##      [,1] [,2] [,3]
## [1,]    2    2   -1
## [2,]    2    4    6
## [3,]   -1    6   25
\end{verbatim}

\hypertarget{getting-eign-values-and-eign-vectors}{%
\subsubsection{Getting eign values and eign
vectors}\label{getting-eign-values-and-eign-vectors}}

\begin{Shaded}
\begin{Highlighting}[]
\NormalTok{eign_X <-}\StringTok{ }\KeywordTok{eigen}\NormalTok{(X)}
\NormalTok{eign_X}
\end{Highlighting}
\end{Shaded}

\begin{verbatim}
## eigen() decomposition
## $values
## [1] 26.601802  4.398198
## 
## $vectors
##           [,1]       [,2]
## [1,] 0.6576043 -0.7533635
## [2,] 0.7533635  0.6576043
\end{verbatim}

\begin{Shaded}
\begin{Highlighting}[]
\NormalTok{eign_Y <-}\StringTok{ }\KeywordTok{eigen}\NormalTok{(Y)}
\NormalTok{eign_Y}
\end{Highlighting}
\end{Shaded}

\begin{verbatim}
## eigen() decomposition
## $values
## [1] 2.660180e+01 4.398198e+00 1.058982e-16
## 
## $vectors
##             [,1]       [,2]       [,3]
## [1,] -0.01856629 -0.6727903  0.7396003
## [2,]  0.25499937 -0.7184510 -0.6471502
## [3,]  0.96676296  0.1765824  0.1849001
\end{verbatim}

\hypertarget{compute-left-right-singular-using-svd-and-compare}{%
\subsubsection{Compute left, right singular using svd() and
compare}\label{compute-left-right-singular-using-svd-and-compare}}

\begin{Shaded}
\begin{Highlighting}[]
\NormalTok{svd_A <-}\StringTok{ }\KeywordTok{svd}\NormalTok{(A)}
\NormalTok{svd_A}
\end{Highlighting}
\end{Shaded}

\begin{verbatim}
## $d
## [1] 5.157693 2.097188
## 
## $u
##            [,1]       [,2]
## [1,] -0.6576043 -0.7533635
## [2,] -0.7533635  0.6576043
## 
## $v
##             [,1]       [,2]
## [1,]  0.01856629 -0.6727903
## [2,] -0.25499937 -0.7184510
## [3,] -0.96676296  0.1765824
\end{verbatim}

\begin{Shaded}
\begin{Highlighting}[]
\CommentTok{#  merge A$u matrix with the eign vector of x}
\NormalTok{compare_X <-}\StringTok{ }\KeywordTok{cbind}\NormalTok{(svd_A}\OperatorTok{$}\NormalTok{u, eign_X}\OperatorTok{$}\NormalTok{vectors)}
\CommentTok{# name the columns}
\KeywordTok{colnames}\NormalTok{(compare_X) <-}\StringTok{ }\KeywordTok{c}\NormalTok{(}\StringTok{'SVDu1'}\NormalTok{, }\StringTok{'SVDu2'}\NormalTok{, }\StringTok{'EVX1=u1'}\NormalTok{, }\StringTok{'EVX2=u2'}\NormalTok{)}

\NormalTok{(compare_X)}
\end{Highlighting}
\end{Shaded}

\begin{verbatim}
##           SVDu1      SVDu2   EVX1=u1    EVX2=u2
## [1,] -0.6576043 -0.7533635 0.6576043 -0.7533635
## [2,] -0.7533635  0.6576043 0.7533635  0.6576043
\end{verbatim}

\begin{Shaded}
\begin{Highlighting}[]
\NormalTok{compare_Y <-}\StringTok{ }\KeywordTok{cbind}\NormalTok{(svd_A}\OperatorTok{$}\NormalTok{v, eign_Y}\OperatorTok{$}\NormalTok{vectors)}
\NormalTok{compare_Y <-}\StringTok{ }\NormalTok{compare_Y[,}\DecValTok{1}\OperatorTok{:}\DecValTok{4}\NormalTok{]}
\KeywordTok{colnames}\NormalTok{(compare_Y) <-}\StringTok{ }\KeywordTok{c}\NormalTok{(}\StringTok{'SVDv1'}\NormalTok{, }\StringTok{'SVDv2'}\NormalTok{, }\StringTok{'EVY1=v1'}\NormalTok{, }\StringTok{'EVY2=v2'}\NormalTok{)}
\NormalTok{(compare_Y)}
\end{Highlighting}
\end{Shaded}

\begin{verbatim}
##            SVDv1      SVDv2     EVY1=v1    EVY2=v2
## [1,]  0.01856629 -0.6727903 -0.01856629 -0.6727903
## [2,] -0.25499937 -0.7184510  0.25499937 -0.7184510
## [3,] -0.96676296  0.1765824  0.96676296  0.1765824
\end{verbatim}

\hypertarget{problem_set2}{%
\subsection{Problem\_set2}\label{problem_set2}}

Using the procedure outlined in section 1 of the weekly handout, write a
function to compute the inverse of a well-conditioned full-rank square
matrix using co-factors. In order to compute the co-factors, you may use
built-in commands to compute the determinant.

Your function should have the following signature: \(B = myinverse(A)\)
where A is a matrix and B is its inverse and \(A \times B = I\). The
off-diagonal elements of I should be close to zero, if not zero.
Likewise, the diagonal elements should be close to 1, if not 1. Small
numerical precision errors are acceptable but the function myinverse
should be correct and must use co-factors and determinant of A to
compute the inverse.

\textbf{Answer:-}

\begin{Shaded}
\begin{Highlighting}[]
\NormalTok{a <-}\StringTok{ }\KeywordTok{matrix}\NormalTok{(}\KeywordTok{c}\NormalTok{(}\DecValTok{2}\NormalTok{,}\DecValTok{4}\NormalTok{,}\OperatorTok{-}\DecValTok{2}\NormalTok{,}
              \DecValTok{1}\NormalTok{,}\OperatorTok{-}\DecValTok{1}\NormalTok{,}\DecValTok{5}\NormalTok{,}
              \DecValTok{3}\NormalTok{,}\DecValTok{3}\NormalTok{,}\DecValTok{5}\NormalTok{), }\DataTypeTok{nrow =} \DecValTok{3}\NormalTok{, }\DataTypeTok{byrow =} \OtherTok{TRUE}\NormalTok{)}
\NormalTok{a}
\end{Highlighting}
\end{Shaded}

\begin{verbatim}
##      [,1] [,2] [,3]
## [1,]    2    4   -2
## [2,]    1   -1    5
## [3,]    3    3    5
\end{verbatim}

\begin{Shaded}
\begin{Highlighting}[]
\NormalTok{getCofactors <-}\StringTok{ }\ControlFlowTok{function}\NormalTok{(M) \{}
  \KeywordTok{stopifnot}\NormalTok{(}\KeywordTok{length}\NormalTok{(}\KeywordTok{unique}\NormalTok{(}\KeywordTok{dim}\NormalTok{(M)))}\OperatorTok{==}\DecValTok{1}\NormalTok{) }\CommentTok{# Check if Matrix = Square}
\NormalTok{  cf <-}\StringTok{ }\NormalTok{M }\CommentTok{# creating a Matrix that has the same Dimensions as M }
  \ControlFlowTok{for}\NormalTok{(i }\ControlFlowTok{in} \DecValTok{1}\OperatorTok{:}\KeywordTok{dim}\NormalTok{(M)[}\DecValTok{1}\NormalTok{])\{}
    \ControlFlowTok{for}\NormalTok{(j }\ControlFlowTok{in} \DecValTok{1}\OperatorTok{:}\KeywordTok{dim}\NormalTok{(M)[}\DecValTok{2}\NormalTok{])\{}
\NormalTok{      cf[i,j] <-}\StringTok{ }\NormalTok{(}\KeywordTok{det}\NormalTok{(M[}\OperatorTok{-}\NormalTok{i,}\OperatorTok{-}\NormalTok{j])}\OperatorTok{*}\NormalTok{(}\OperatorTok{-}\DecValTok{1}\NormalTok{)}\OperatorTok{^}\NormalTok{(i}\OperatorTok{+}\NormalTok{j)) }\CommentTok{# overwriting the Values of cf Matrix with cofactors}
\NormalTok{    \}}
\NormalTok{  \}}
  \KeywordTok{return}\NormalTok{(cf) }\CommentTok{# output of cofactors matrix}
\NormalTok{\}}

\NormalTok{a_cofactor <-}\StringTok{ }\KeywordTok{getCofactors}\NormalTok{(a)}
\NormalTok{a_cofactor}
\end{Highlighting}
\end{Shaded}

\begin{verbatim}
##      [,1] [,2] [,3]
## [1,]  -20   10    6
## [2,]  -26   16    6
## [3,]   18  -12   -6
\end{verbatim}

\begin{Shaded}
\begin{Highlighting}[]
\NormalTok{myinverse <-}\StringTok{ }\ControlFlowTok{function}\NormalTok{(A)\{}
\NormalTok{    det_A =}\StringTok{ }\KeywordTok{det}\NormalTok{(A) }\CommentTok{# get the determinant of matrix  A}
\NormalTok{    inverse <-}\StringTok{ }\NormalTok{(}\DecValTok{1}\OperatorTok{/}\NormalTok{det_A) }\OperatorTok{*}\StringTok{ }\KeywordTok{t}\NormalTok{(}\KeywordTok{getCofactors}\NormalTok{(A)) }\CommentTok{# inverse should be determinant inverse multiplied by the transpose of the cofactor matrix}
    \KeywordTok{return}\NormalTok{(inverse)}
\NormalTok{\}}

\NormalTok{B <-}\StringTok{ }\KeywordTok{myinverse}\NormalTok{(a)}
\NormalTok{B}
\end{Highlighting}
\end{Shaded}

\begin{verbatim}
##            [,1]      [,2] [,3]
## [1,]  1.6666667  2.166667 -1.5
## [2,] -0.8333333 -1.333333  1.0
## [3,] -0.5000000 -0.500000  0.5
\end{verbatim}


\end{document}
